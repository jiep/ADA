\documentclass[12pt,a4paper,twoside,openright,titlepage,final]{article}
\usepackage{fontspec}
\usepackage{amsmath}
\usepackage{amsfonts}
\usepackage{amssymb}
\usepackage{makeidx}
\usepackage{graphicx}
\usepackage[hidelinks,unicode=true]{hyperref}
\usepackage[spanish,es-nodecimaldot,es-lcroman,es-tabla,es-noshorthands]{babel}
\usepackage[left=3cm,right=2cm, bottom=4cm]{geometry}
\usepackage{natbib}
\usepackage{microtype}
\usepackage{ifdraft}
\usepackage{verbatim}
\usepackage[nottoc]{tocbibind}
\usepackage{pdflscape}
\usepackage{fancyvrb}
\usepackage[obeyDraft]{todonotes}
\ifdraft{
	\usepackage{draftwatermark}
	\SetWatermarkText{BORRADOR}
	\SetWatermarkScale{0.7}
	\SetWatermarkColor{red}
}{}
\usepackage{booktabs}
\usepackage{longtable}
\usepackage{calc}
\usepackage{array}
\usepackage{caption}
\usepackage{subfigure}
\usepackage{footnote}
\usepackage{url}
\usepackage[titletoc]{appendix}

\setsansfont[Ligatures=TeX]{texgyreadventor}
\setmainfont[Ligatures=TeX]{texgyrepagella}
\setmonofont{FreeMono}

\usetikzlibrary{decorations.pathreplacing}

\input{portada}

\author{José Ignacio Escribano}

\title{}

\setlength{\parindent}{0pt}

\begin{document}

\pagenumbering{alph}
\setcounter{page}{1}

\portada{Trabajo final}{Análisis de Datos Avanzados}{Análisis de series temporales}{José Ignacio Escribano}{Móstoles}

\tableofcontents
\thispagestyle{empty}
\newpage

\listoffigures
\thispagestyle{empty}
\newpage

\listoftables
\thispagestyle{empty}
\newpage

\pagenumbering{arabic}
\setcounter{page}{1}

\section{Introducción}

En este caso práctico utilizaremos la metodología Box-Jenkins para analizar dos series temporales. La primera es el índice de empleo de un determinado país, y la segunda es el volumen de ventas mensual de puros de una empresa tabacalera. En ambos casos, se trata de obtener un modelo que se ajuste lo máximo posible a la serie temporal.\\

La metodología Box-Jenkins recoge los pasos necesarios para obtener el modelo más adecuado de serie temporal:

\begin{enumerate}
	\item Especificación inicial: consiste en determinar el orden de integración de la serie temporal y naturaleza de diferencias que se requerirán para convertir en estacionaria la serie temporal. En este paso se usa el análisis gráfico de la serie, además de los correlogramas simple y parcial de la serie. 
	Una vez hecho lo anterior, habrá que decidir los órdenes de los polinomios autorregresivo y de medias móviles. De nuevo, se hará uso del correlograma simple y parcial de la serie. La Tabla~\ref*{tbl:modelos} recoge las principales características de la función de autocorrelación y de autocorrelación parcial de los principales modelos estacionarios. 
	
	\begin{table}[htbp!]
		\centering
		\caption{Principales características de la función de autocorrelación y de autocorrelación parcial de los principales modelos estacionarios}
		\label{tbl:modelos}
		\begin{tabular}{@{}ccc@{}}
			\toprule
			\textbf{Modelo} & \textbf{\begin{tabular}[c]{@{}c@{}}Función de \\ autocorrelación\end{tabular}}                            & \textbf{\begin{tabular}[c]{@{}c@{}}Función de \\ autocorrelación parcial\end{tabular}}                      \\ \midrule
			AR(p)           & \begin{tabular}[c]{@{}c@{}}Decrecimiento rápido hacia cero, \\ sin llegar a anularse\end{tabular}         & \begin{tabular}[c]{@{}c@{}}$p$ primera autocorrelaciones distintas \\ de cero, y el resto cero\end{tabular} \\
			MA(q)           & \begin{tabular}[c]{@{}c@{}}$q$ primeras autocorrelaciones \\ significativas, y el resto cero\end{tabular} & \begin{tabular}[c]{@{}c@{}}Decrecimiento rápido hacia cero,\\ sin llegar a anularse\end{tabular}            \\
			ARMA(p,q)       & \begin{tabular}[c]{@{}c@{}}Decrecimiento rápido hacia cero, \\ sin llegar a anularse\end{tabular}           & \begin{tabular}[c]{@{}c@{}}Decrecimiento rápido hacia cero,\\ sin llegar a anularse\end{tabular}            \\ \bottomrule
		\end{tabular}
	\end{table}
	
	\item Estimación: en este paso, se procede a estimar los modelos propuestos, normalmente mediante máxima verosimilitud o mínimos cuadrados no lineales. 
	
	\item Chequeo o validación: en este paso, se validan los posibles modelos y se escoge el que parezca más adecuado para describir la serie temporal.
	
	\item Utilización del modelo: el modelo escogido se puede utilizar para predecir futuros valores de la serie.
\end{enumerate}

\section{Resolución de las series temporales}

A continuación, aplicamos la metodología Box-Jenkins para obtener un modelo que se adecue a cada una de las series temporales planteadas. 

\subsection{Índice de empleo de un determinado país}

La primera serie temporal es el índice de empleo de un determinado país. La serie está corregida de estacionalidad y tiene frecuencia trimestral. El período muestral abarca desde el primer trimestre del año 1962 hasta el cuarto trimestre del año 1994.\\

 

\subsection{Venta de cigarros puros de una empresa tabacalera}

La segunda serie temporal es el volumen de ventas mensual de puros de una empresa tabacalera. El período de la serie abarca desde enero de 1989 hasta diciembre de 1996.

\section{Conclusiones}


\end{document}